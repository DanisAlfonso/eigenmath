\input{preamble}

\section*{Arc length}

Let $g(t)$ be a parametric function that draws a curve in $\mathbb R^n$.
The arc length from $g(a)$ to $g(b)$ is given by
\begin{equation*}
\int_a^b|g'(t)|\,dt
\end{equation*}

where $|g'(t)|$ is the length of the tangent vector at $g(t)$.

\bigskip
Example 1.
Find the length of the curve $y=x^2$ from $x=0$ to $x=1$.

{\color{blue}
\begin{verbatim}
g = (t,t^2)
defint(abs(d(g,t)),t,0,1)
\end{verbatim}}

$
\tfrac{1}{2}\;5^{1/2}
-\tfrac{1}{4}\log(2)
+\tfrac{1}{4}\log(2\;5^{1/2}+4)
$

{\color{blue}
\begin{verbatim}
float
\end{verbatim}}

$1.47894$

\bigskip
As expected, the result is greater than $\sqrt2\approx1.414$,
the length of a straight line from $(0,0)$ to $(1,1)$.

\bigskip
The following script does a discrete computation of the arc length
by dividing the curve into 100 pieces.

{\color{blue}
\begin{verbatim}
g(t) = (t,t^2)
h(k) = abs(g(k/100.0) - g((k-1)/100.0))
sum(k,1,100,h(k))
\end{verbatim}}

$1.47894$

\bigskip
As expected, the discrete result matches the analytic result.

\bigskip
Example 2.
Find the length of the curve $y=x^{3/2}$ from the origin to
$x=\tfrac{4}{3}$.

{\color{blue}
\begin{verbatim}
g = (t,t^(3/2))
defint(abs(d(g,t)),t,0,4/3)
\end{verbatim}}

$\displaystyle \tfrac{56}{27}$

\end{document}
