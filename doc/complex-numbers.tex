\input{preamble}

\section*{Complex numbers}

Symbol \verb$i$ is initialized to $\sqrt{-1}$.

\bigskip

Complex quantities can be entered in either rectangular or polar form.

{\color{blue}
\begin{verbatim}
a + i b
\end{verbatim}
}

$\displaystyle a+ib$

{\color{blue}
\begin{verbatim}
exp(1/3 i pi)
\end{verbatim}
}

$\displaystyle \exp\left(\tfrac{1}{3}i\pi\right)$

\bigskip

Converting a complex number to rectangular or polar coordinates causes
simplification of mixed forms.

{\color{blue}
\begin{verbatim}
A = 1 + i
B = sqrt(2) exp(1/4 i pi)
A - B
\end{verbatim}
}

$\displaystyle 1+i-2^{1/2}\exp\left(\tfrac{1}{4}i\pi\right)$

{\color{blue}
\begin{verbatim}
rect(last)
\end{verbatim}
}

$\displaystyle 0$

\bigskip

Rectangular complex quantities, when raised to a power, are multiplied out.

{\color{blue}
\begin{verbatim}
(a + i b)^2
\end{verbatim}
}

$\displaystyle a^2-b^2+2iab$

\bigskip

When $a$ and $b$ are numerical and the power is negative, the evaluation is done as follows.
\begin{equation*}
(a+ib)^{-n}
=\left(\frac{a-ib}{(a+ib)(a-ib)}\right)^n=
\left(\frac{a-ib}{a^2+b^2}\right)^n
\end{equation*}

Here are a few examples.

{\color{blue}
\begin{verbatim}
1/(2 - i)
\end{verbatim}
}

$\displaystyle \tfrac{2}{5}+\tfrac{1}{5}i$

{\color{blue}
\begin{verbatim}
(-1 + 3 i)/(2 - i)
\end{verbatim}
}

$\displaystyle -1+i$

\bigskip

The absolute value of a complex number returns its magnitude.

{\color{blue}
\begin{verbatim}
abs(3 + 4 i)
\end{verbatim}
}

$\displaystyle 5$

\bigskip

The imaginary unit can be changed from $i$ to $j$
by defining $j=\sqrt{-1}$.

{\color{blue}
\begin{verbatim}
j = sqrt(-1)
sqrt(-4)
\end{verbatim}
}

$\displaystyle 2j$

\end{document}
