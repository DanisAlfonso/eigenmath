\documentclass[12pt]{article}
\usepackage[margin=1in]{geometry}
\usepackage{amsmath}
\usepackage{amssymb} % \mathbb
\usepackage{graphicx}
\usepackage{tikz}
\usepackage{xcolor}
\usepackage{menukeys}
\parindent=0pt
\title{Eigenmath Manual}
\author{9634295@gmail.com}
\begin{document}
\maketitle
\newpage
\tableofcontents
\newpage

Commands are entered in the following field.

\begin{center}
\begin{tikzpicture}
\node at (0,0) {\includegraphics[scale=0.25]{screenshot.png}};
\draw[red,thick] (3,-2.6) ellipse (3.5cm and 0.5cm);
\end{tikzpicture}
\end{center}

Multiple commands can be put together in a script.
Scripts are run by clicking the Run button.

\begin{center}
\begin{tikzpicture}
\node at (0,0) {\includegraphics[scale=0.25]{screenshot.png}};
\draw (-3,0) node {Scripts go here};
\end{tikzpicture}
\end{center}

After a script runs, all of the results are available in command mode.

\iffalse
\bigskip
To print or copy results, click in the result field.
Then press \cmd$\,$P to print, \cmd$\,$C to copy to the clipboard.
\fi

\bigskip
Note: Eigenmath expects Times New Roman and Times New Roman Italic fonts
to be the standard macOS fonts that include special symbols and Greek letters.
See the following link for correcting font problems.

\bigskip
{\footnotesize\verb$support.apple.com/guide/font-book/restore-fonts-that-came-with-your-mac-fb34862/mac$}

\newpage

\section{Introduction}

In the following examples, user input is shown in blue.
Results are shown in black.

\bigskip
Example 1. Compute $212^{17}$.

{\color{blue}
\begin{verbatim}
212^17
\end{verbatim}}

$3529471145760275132301897342055866171392$

\bigskip
Example 2. Compute $212^{17}$ and save as $N$,
then show the value of $N$.

{\color{blue}
\begin{verbatim}
N = 212^17
N
\end{verbatim}}

$N=3529471145760275132301897342055866171392$

\bigskip
Example 3. Compute the 17th root of $N$.

{\color{blue}
\begin{verbatim}
N^(1/17)
\end{verbatim}}

$212$

\iffalse
\bigskip
Note: The above examples were inspired by the following passage from
Vladimir Nabokov's autobiography ``Speak, Memory.''

\begin{quote}
A foolish tutor had explained logarithms to me much too early, and I had
read (in a British publication, the {\it Boy's Own Paper}, I believe)
about a certain Hindu calculator who in exactly two seconds could find the
seventeenth root of, say,
3529471145760275132301897342055866171392
(I am not sure I have got this right; anyway the root was 212).
\end{quote}
\fi

\newpage

\section{Syntax}

\begin{tabular}{clll}
{\it Math} & & {\it Eigenmath} & {\it Comment}
\\
\\
$a=b$ & & \verb$a == b$ & {\it test for equality}
\\[1ex]
$-a$ & & {\tt -a} & {\it negation}
\\[1ex]
$a+b$ & & {\tt a+b} & {\it addition}
\\[1ex]
$a-b$ & & {\tt a-b} & {\it subtraction}
\\[1ex]
$ab$ & & {\tt a b} & {\it multiplication, also} \verb$a*b$
\\
\\
$\displaystyle\frac{a}{b}$ & & {\tt a/b} & {\it division}
\\
\\
$\displaystyle\frac{a}{bc}$ & & {\tt a/b/c} & {\it division is left-associative}
\\
\\
$a^2$ & & {\tt a{\char94}2} & {\it power}
\\
\\
$\sqrt{a}$ & & \verb$sqrt(a)$ & {\it square root, also} \verb$a^(1/2)$
\\
\\
$a\,(b+c)$ & & {\tt a (b+c)} & {\it space is required}
\\
\\
$f(a)$ & & {\tt f(a)} & {\it function}
\\
\\
$\begin{pmatrix}a\\ b\\ c\end{pmatrix}$ & & {\tt (a,b,c)} & {\it vector}
\\
\\
$\begin{pmatrix}a&b\\ c&d\end{pmatrix}$ & & {\tt ((a,b),(c,d))} & {\it matrix}
\\
\\
$F^1{}_2$ & & {\tt F[1,2]} & {\it tensor component access}
\\
\\
 & & \verb$"hello, world"$ & {\it string literal}
\\
\\
$\pi$ & & {\tt pi} &
\\[1ex]
$e$ && {\tt exp(1)} & {\it natural number}
\end{tabular}

\newpage

Arithmetic operators have the expected precedence of
multiplication and division before addition and subtraction.
Subexpressions in parentheses have highest precedence.

\bigskip
Parentheses are required around negative exponents.
For example,

{\color{blue}
\begin{verbatim}
10^(-3)
\end{verbatim}
}

instead of

{\color{blue}
\begin{verbatim}
10^-3
\end{verbatim}
}

The reason for this is that the binding of the negative sign is not always obvious.
For example, consider

{\color{blue}
\begin{verbatim}
x^-1/2
\end{verbatim}
}

It is not clear whether the exponent should be $-1$ or $-1/2$.
Hence the following syntax is required.

{\color{blue}
\begin{verbatim}
x^(-1/2)
\end{verbatim}
}

In general, parentheses are always required when the exponent
is an expression.
For example, \verb$x^1/2$ is evaluated as $(x^1)/2$ which
is probably not the desired result.

{\color{blue}
\begin{verbatim}
x^1/2
\end{verbatim}
}

$\displaystyle \tfrac{1}{2}x$

\bigskip

Using \verb$x^(1/2)$ yields the desired result.

{\color{blue}
\begin{verbatim}
x^(1/2)
\end{verbatim}
}

$\displaystyle x^{1/2}$

\newpage

\section{Symbols}

Symbols are defined with an equals sign.

{\color{blue}
\begin{verbatim}
N = 212^17
\end{verbatim}
}

No result is printed when a symbol is defined.
To see the value of a symbol, just evaluate it.

{\color{blue}
\begin{verbatim}
N
\end{verbatim}
}

$\displaystyle N=3529471145760275132301897342055866171392$

\bigskip

Symbols can have more that one letter.
Everything after the first letter is displayed as a subscript.

{\color{blue}
\begin{verbatim}
NA = 6.02214 10^23
NA
\end{verbatim}
}

$\displaystyle N_A=6.02214\times10^{23}$

\bigskip

A symbol can be the name of a Greek letter.

{\color{blue}
\begin{verbatim}
xi = 1/2
xi
\end{verbatim}
}

$\displaystyle \xi=\tfrac{1}{2}$

\bigskip

Greek letters can appear in subscripts.

{\color{blue}
\begin{verbatim}
Amu = 2.0
Amu
\end{verbatim}
}

$\displaystyle A_\mu=2.0$

\bigskip

The following example shows how a symbol is scanned to find Greek letters.

{\color{blue}
\begin{verbatim}
alphamunu = 1
alphamunu
\end{verbatim}
}

$\displaystyle \alpha_{\mu\nu}=1$

\bigskip

Symbol definitions are evaluated serially until a terminal symbol is reached.
The following example sets $A=B$ followed by $B=C$.
Then when $A$ is evaluated, the result is $C$.

{\color{blue}
\begin{verbatim}
A = B
B = C
A
\end{verbatim}
}

$\displaystyle A=C$

\bigskip

Although $A=C$ is printed,
inside the program the binding of $A$ is still $B$, as can be seen with
the \verb$binding$ function.

{\color{blue}
\begin{verbatim}
binding(A)
\end{verbatim}
}

$\displaystyle B$

\bigskip

The \verb$quote$ function returns its argument unevaluated
and can be used to clear a symbol.
The following example clears $A$ so that its evaluation goes back to
being $A$ instead of $C$.

{\color{blue}
\begin{verbatim}
A = quote(A)
A
\end{verbatim}
}

$\displaystyle A$

\newpage

\section{Function definitions}

The syntax for defining functions is
\begin{center}
{\it function-name} ( {\it arg-list} ) = {\it expr}
\end{center}
where {\it arg-list} is a comma separated list of zero to nine symbols that receive arguments.
Unlike symbol definitions, {\it expr} is not evaluated when {\it function-name} is defined.
Instead, {\it expr} is evaluated when {\it function-name} is used in a subsequent computation.
The scope of function arguments is the function definition {\it expr}.

\bigskip
Function definitions cannot be nested.
In other words, function definition {\it expr}
cannot contain another function definition.

\bigskip
The following example defines a sinc function and evaluates it at $\pi/2$.

{\color{blue}
\begin{verbatim}
f(x) = sin(x)/x
f(pi/2)
\end{verbatim}}

$\displaystyle \frac{2}{\pi}$

\bigskip
After a user function is defined, {\it expr} can be recalled using the
\verb$binding$ function.

{\color{blue}
\begin{verbatim}
binding(f)
\end{verbatim}}

$\displaystyle \frac{\sin(x)}{x}$

\bigskip
To define a local symbol for use inside {\it expr}, extend the argument list.
In the following example, argument $y$ is used as a local symbol.
Note that function $L$ is called without supplying an argument for $y$.

{\color{blue}
\begin{verbatim}
L(f,n,y) = eval(exp(y) / n! d(exp(-y) y^n, y, n), y, f)
L(cos(x),2)
\end{verbatim}
}

$\displaystyle \tfrac{1}{2}\cos(x)^2-2\cos(x)+1$

\bigskip
Sometimes it is necessary to evauate an argument at a particular value.
Use \verb$eval$ to evaluate function arguments inside {\it expr}.

{\color{blue}
\begin{verbatim}
h(f,x,a) = abs(eval(f,x,a))
h(cos(y),y,0)
\end{verbatim}}

$1$

\newpage

\section{Arithmetic}

Big integer arithmetic is used so that numerical values can
exceed machine size.

{\color{blue}
\begin{verbatim}
2^64
\end{verbatim}
}

$\displaystyle 18446744073709551616$

{\color{blue}
\begin{verbatim}
212^17
\end{verbatim}
}

$\displaystyle 3529471145760275132301897342055866171392$

\bigskip

Rational number arithmetic is used by default.

{\color{blue}
\begin{verbatim}
1/2 + 1/3
\end{verbatim}
}

$\displaystyle \tfrac{5}{6}$

\bigskip

Floating point arithmetic can also be used.

{\color{blue}
\begin{verbatim}
1/2 + 1/3.0
\end{verbatim}
}

$\displaystyle 0.833333$

\bigskip

An integer or rational number result can be converted to a floating
point value by entering \verb$float$.

{\color{blue}
\begin{verbatim}
212^17
\end{verbatim}
}

$\displaystyle 3529471145760275132301897342055866171392$

{\color{blue}
\begin{verbatim}
float
\end{verbatim}
}

$\displaystyle 3.52947\times10^{39}$

\bigskip
The following example shows how to enter a floating point value
using scientific notation.

{\color{blue}
\begin{verbatim}
epsilon = 1.0 10^(-6)
epsilon
\end{verbatim}
}

$\displaystyle \varepsilon=1.0\times10^{-6}$

\newpage

\section{Complex numbers}

Symbol \verb$i$ is initialized to $\sqrt{-1}$.

\bigskip

Complex quantities can be entered in either rectangular or polar form.

{\color{blue}
\begin{verbatim}
a + i b
\end{verbatim}
}

$\displaystyle a+ib$

{\color{blue}
\begin{verbatim}
exp(1/3 i pi)
\end{verbatim}
}

$\displaystyle \exp\left(\tfrac{1}{3}i\pi\right)$

\bigskip

Converting a complex number to rectangular or polar coordinates causes
simplification of mixed forms.

{\color{blue}
\begin{verbatim}
A = 1 + i
B = sqrt(2) exp(1/4 i pi)
A - B
\end{verbatim}
}

$\displaystyle 1+i-2^{1/2}\exp\left(\tfrac{1}{4}i\pi\right)$

{\color{blue}
\begin{verbatim}
rect(last)
\end{verbatim}
}

$\displaystyle 0$

\bigskip

Rectangular complex quantities, when raised to a power, are multiplied out.

{\color{blue}
\begin{verbatim}
(a + i b)^2
\end{verbatim}
}

$\displaystyle a^2-b^2+2iab$

\bigskip

When $a$ and $b$ are numerical and the power is negative, the evaluation is done as follows.
\begin{equation*}
(a+ib)^{-n}
=\left(\frac{a-ib}{(a+ib)(a-ib)}\right)^n=
\left(\frac{a-ib}{a^2+b^2}\right)^n
\end{equation*}

Here are a few examples.

{\color{blue}
\begin{verbatim}
1/(2 - i)
\end{verbatim}
}

$\displaystyle \tfrac{2}{5}+\tfrac{1}{5}i$

{\color{blue}
\begin{verbatim}
(-1 + 3 i)/(2 - i)
\end{verbatim}
}

$\displaystyle -1+i$

\bigskip

The absolute value of a complex number returns its magnitude.

{\color{blue}
\begin{verbatim}
abs(3 + 4 i)
\end{verbatim}
}

$\displaystyle 5$

\bigskip

The imaginary unit can be changed from $i$ to $j$
by defining $j=\sqrt{-1}$.

{\color{blue}
\begin{verbatim}
j = sqrt(-1)
sqrt(-4)
\end{verbatim}
}

$\displaystyle 2j$

\newpage

\section{Draw}

\verb$draw(f,x)$ draws a graph of function $f$ of $x$.

{\color{blue}
\begin{verbatim}
draw(x^2,x)
\end{verbatim}}

\begin{center}
\includegraphics[scale=0.4]{parabola1.png}
\end{center}

The vectors \verb$xrange$ and \verb$yrange$ control the scale of the graph.

{\color{blue}
\begin{verbatim}
xrange = (-1,1)
yrange = (0,2)
draw(x^2)
\end{verbatim}}

\begin{center}
\includegraphics[scale=0.4]{parabola2.png}
\end{center}

Parametric drawing occurs when a function returns a vector.
The vector \verb$trange$ controls the parametric range.
The default is \verb$trange=(-pi,pi)$.
In the following example, \verb$draw$ varies \verb$theta$
over the default range $-\pi$ to $+\pi$.

{\color{blue}
\begin{verbatim}
xrange = (-10,10)
yrange = (-10,10)
f = 5 (cos(theta),sin(theta))
draw(f,theta)
\end{verbatim}}

\begin{center}
\includegraphics[scale=0.4]{circle1.png}
\end{center}

In the following example, \verb$trange$ is reduced
to draw a quarter circle instead of a full circle.

{\color{blue}
\begin{verbatim}
trange = (0,pi/2)
f = 5 (cos(theta),sin(theta))
draw(f,theta)
\end{verbatim}}

\begin{center}
\includegraphics[scale=0.4]{circle2.png}
\end{center}

Draw a lemniscate.

{\color{blue}
\begin{verbatim}
trange = (-pi,pi)
X = cos(t) / (1 + sin(t)^2)
Y = sin(t) cos(t) / (1 + sin(t)^2)
f = 5 (X,Y)
draw(f,t)
\end{verbatim}}

\begin{center}
\includegraphics[scale=0.4]{lemniscate.png}
\end{center}

Draw a cardioid.

{\color{blue}
\begin{verbatim}
r = (1 + cos(t)) / 2
u = (cos(t),sin(t))
f = r u
xrange = (-1,1)
yrange = (-1,1)
trange = (0,2pi)
draw(f,t)
\end{verbatim}}

\begin{center}
\includegraphics[scale=0.4]{cardioid.png}
\end{center}

\newpage

\section{Linear algebra}

\verb$dot(a,b,...)$ returns the inner product of vectors, matrices,
and higher rank tensors.
Also known as the matrix product.
Arguments are evaluated from right to left
for maximum efficiency when the rightmost argument is a vector.

\bigskip

Example 1. Compute the product $AX$ for
\begin{equation*}
A=\begin{pmatrix}a_{11}&a_{12}\\a_{21}&a_{22}\end{pmatrix},
\quad
X=\begin{pmatrix}x_1\\x_2\end{pmatrix}
\end{equation*}

{\color{blue}
\begin{verbatim}
A = ((a11,a12),(a21,a22))
X = (x1,x2)
dot(A,X)
\end{verbatim}
}

$\displaystyle
\begin{bmatrix}
a_{11}x_1+a_{12}x_2
\\[1ex]
a_{21}x_1+a_{22}x_2
\end{bmatrix}
$

\bigskip

Example 2. Solve for vector $X$ in $AX=B$.

{\color{blue}
\begin{verbatim}
A = ((3,7),(1,-9))
B = (16,-22)
X = dot(inv(A),B)
X
\end{verbatim}
}

$\displaystyle
X=
\begin{bmatrix}
-\frac{5}{17}
\\[1ex]
\frac{41}{17}
\end{bmatrix}
$

\bigskip

Example 3. Show that
\begin{equation*}
A^{-1}=\frac{\operatorname{adj}A}{\operatorname{det}A}
\end{equation*}

{\color{blue}
\begin{verbatim}
A = ((a,b),(c,d))
inv(A) == adj(A) / det(A)
\end{verbatim}
}

$1$

\iffalse

\bigskip

Square brackets are used for component access.
Index numbering starts with 1.

{\color{blue}
\begin{verbatim}
A = ((a,b),(c,d))
A[1,2] = -A[1,1]
A
\end{verbatim}
}

$\displaystyle
\begin{bmatrix}
a & -a
\\[1ex]
c & d
\end{bmatrix}
$

\bigskip

Sometimes a calculation will be simpler if it can be reorganized to use
\verb$adj$ instead of \verb$inv$.
The main idea is to try to prevent the determinant from appearing as a
divisor.
For example, suppose for matrices $A$ and $B$ you want to show that
\begin{equation*}
{A}-{B}^{-1}=0
\end{equation*}

Depending on the complexity of $\mathop{\rm det}B$, the software
may not be able to find a simplification that yields zero.
A trick is to multiplying by $\operatorname{det}B$ and try
\begin{equation*}
A\operatorname{det}B-\operatorname{adj}B=0
\end{equation*}

\fi

\newpage

\section{Component arithmetic}

Tensor plus scalar adds scalar to each tensor component.

{\color{blue}
\begin{verbatim}
A = ((a,b),(c,d))
A + 10
\end{verbatim}}

$\displaystyle
\begin{bmatrix}
a+10 & b+10
\\[1ex]
c+10 & d+10
\end{bmatrix}
$

\bigskip
The product of two tensors is the Hadamard (element-wise) product.

{\color{blue}
\begin{verbatim}
A = ((a,b),(c,d))
A A
\end{verbatim}}

$\displaystyle
\begin{bmatrix}
a^2 & b^2
\\[1ex]
c^2 & d^2
\end{bmatrix}
$

\bigskip

Tensor raised to a power raises each component to the power.

{\color{blue}
\begin{verbatim}
A = ((a,b),(c,d))
A^2
\end{verbatim}}

$\displaystyle
\begin{bmatrix}
a^2 & b^2
\\[1ex]
c^2 & d^2
\end{bmatrix}
$

\newpage

\section{Quantum computing}

A quantum computer can be simulated by applying rotations to a
unit vector
$u\in\mathbb{C}^{2^n}$ where $\mathbb{C}$ is the set of complex numbers
and $n$ is the number of qubits.
The dimension is $2^n$ because a register with $n$ qubits
has $2^n$ eigenstates.
(Recall that an eigenstate is the output of a quantum computer.)
Quantum operations are ``rotations'' because they preserve $|u|=1$.
Mathematically, a rotation of $u$ is equivalent to the product $Ru$
where $R$ is a $2^n\times2^n$ matrix.

\bigskip
Eigenstates $|j\rangle$ are represented by the following vectors.
(Each vector has $2^n$ elements.)
\begin{align*}
&|0\rangle=(1,0,0,\dots,0)
\\
&|1\rangle=(0,1,0,\ldots,0)
\\
&|2\rangle=(0,0,1,\ldots,0)
\\
&\vdots
\\
&|2^n-1\rangle=(0,0,0,\ldots,1)
\end{align*}

A quantum computer algorithm is a sequence of rotations
applied to the initial state $|0\rangle$.
(The sequence could be combined into a single rotation
by associativity of matrix multiplication.)
Let $\psi_f$ be the final state of the quantum computer
after all the rotations have been applied.
Like any other state, $\psi_f$ is a linear combination of eigenstates.
\begin{equation*}
\psi_f=\sum_{j=0}^{2^n-1}c_j|j\rangle,
\quad
c_j\in\mathbb C,
\quad
|\psi_f|=1
\end{equation*}

The last step is to measure $\psi_f$ and get a result.
Measurement rotates $\psi_f$ to an eigenstate $|j\rangle$.
The measurement result is $|j\rangle$.
The probability $P_j$ of getting a specific result $|j\rangle$ is
\begin{equation*}
P_j=|c_j|^2=c_jc_j^*
\end{equation*}

Note that if $\psi_f$ is already an eigenstate then no rotation occurs.
(The probability of observing a different eigenstate is zero.)
Since the measurement result is always an eigenstate,
the coefficients $c_j$ cannot be observed.
However, the same calculation can be run multiple times
to obtain a probability distribution of results.
The probability distribution is an estimate
of $|c_j|^2$ for each $|j\rangle$ in $\psi_f$.

\bigskip
Unlike a real quantum computer, in a simulation
the final state $\psi_f$, or any other state, is available for inspection.
Hence there is no need to simulate the measurement process.
The probability distribution of the result can be computed directly as
\begin{equation*}
P=\psi_f\,\psi_f^*
\end{equation*}

where $\psi_f\,\psi_f^*$
is the Hadamard (element-wise) product of vector $\psi_f$ and its complex conjugate.
Result $P$ is a vector such that $P_j$ is the
probability of eigenstate $|j\rangle$ and
\begin{equation*}
\sum_{j=0}^{2^n-1} P_j=1
\end{equation*}

Note: Eigenmath index numbering begins with 1 hence
\verb$P[1]$ is the probability of $|0\rangle$,
\verb$P[2]$ is the probability of $|1\rangle$, etc.

\bigskip
The Eigenmath function
$\text{\tt rotate}(u,s,k,\ldots)$
rotates vector $u$ and returns the result.
Vector $u$ is required to have $2^n$ elements where $n$ is an
integer from 1 to 15.
Arguments $s,k,\ldots$ are a sequence of rotation codes
where $s$ is an upper case letter and $k$ is a qubit number
from 0 to $n-1$.
Rotations are evaluated from left to right.
The available rotation codes are

\begin{center}
\begin{tabular}{ll}
$C,k$ & Control prefix
\\
$H,k$ & Hadamard
\\
$P,k,\phi$ & Phase modifier (use $\phi=\tfrac{1}{4}\pi$ for $T$ rotation)
\\
$Q,k$ & Quantum Fourier transform
\\
$V,k$ & Inverse quantum Fourier transform
\\
$W,k,j$ & Swap bits
\\
$X,k$ & Pauli X
\\
$Y,k$ & Pauli Y
\\
$Z,k$ & Pauli Z
\end{tabular}
\end{center}

Control prefix $C,k$ modifies the next rotation code so that it
is a controlled rotation with $k$ as the control qubit.
Use two or more prefixes to specify multiple control qubits.
For example, $C,k,C,j,X,m$ is a Toffoli rotation.
Fourier rotations $Q,k$ and $V,k$ are applied to qubits 0 through $k$.
($Q$ and $V$ ignore any control prefix.)

\bigskip
List of $\text{\tt rotate}(u,s,k,\ldots)$ error codes:
\begin{itemize}
\item[1] Argument $u$ is not a vector or does not have $2^n$ elements where $n=1,2,\ldots,15$.
\item[2] Unexpected end of argument list (i.e., missing argument).
\item[3] Bit number format error or range error.
\item[4] Unknown rotation code.
\end{itemize}

Example:
Verify the following truth table for quantum operator CNOT
where qubit 0 is the control and qubit 1 is the target.
(Target is inverted when control is set.)
\begin{center}
\begin{tabular}{ccc}
Target & Control & Output \\
0 & 0 & 00 \\
0 & 1 & 11 \\
1 & 0 & 10 \\
1 & 1 & 01
\end{tabular}
\end{center}

{\color{blue}
\begin{verbatim}
U(psi) = rotate(psi,C,0,X,1) -- CNOT, control 0, target 1

ket00 = (1,0,0,0)
ket01 = (0,1,0,0)
ket10 = (0,0,1,0)
ket11 = (0,0,0,1)

U(ket00) == ket00
U(ket01) == ket11
U(ket10) == ket10
U(ket11) == ket01
\end{verbatim}
}

Here are some useful Eigenmath code snippets for setting up a simulation
and computing the result.

\bigskip
1. Initialize $\psi=|0\rangle$.
{\color{blue}
\begin{verbatim}
n = 4           -- number of qubits (example)
N = 2^n         -- number of eigenstates
psi = zero(N)
psi[1] = 1
\end{verbatim}
}

2. Compute the probability distribution for state $\psi$.
{\color{blue}
\begin{verbatim}
P = psi conj(psi)
\end{verbatim}
}

Hence
\begin{align*}
&\text{\tt P[1]}=\text{probability that $|0\rangle$ will be the result}
\\
&\text{\tt P[2]}=\text{probability that $|1\rangle$ will be the result}
\\
&\text{\tt P[3]}=\text{probability that $|2\rangle$ will be the result}
\\
&\vdots
\\
&\text{\tt P[N]}=\text{probability that $|N-1\rangle$ will be the result}
\end{align*}

3. (Only for macOS) Draw a probability distribution.
{\color{blue}
\begin{verbatim}
xrange = (0,N)
yrange = (0,1)
draw(P[ceiling(x)],x)
\end{verbatim}
}

4. Compute an expectation value.
{\color{blue}
\begin{verbatim}
sum(k,1,N, (k - 1) P[k])
\end{verbatim}
}

5. Make the high order qubit ``don't care.''
{\color{blue}
\begin{verbatim}
for(k,1,N/2, P[k] = P[k] + P[k + N/2])
\end{verbatim}
}

Hence for $N=16$
\begin{align*}
&\text{\tt P[1]}=\text{probability that the result will be $|0\rangle$ or $|8\rangle$}
\\
&\text{\tt P[2]}=\text{probability that the result will be $|1\rangle$ or $|9\rangle$}
\\
&\text{\tt P[3]}=\text{probability that the result will be $|2\rangle$ or $|10\rangle$}
\\
&\vdots
\\
&\text{\tt P[8]}=\text{probability that the result will be $|7\rangle$ or $|15\rangle$}
\end{align*}

\newpage

\section{Derivative}

\verb$d(f,x)$ returns the derivative of $f$ with respect to $x$.

{\color{blue}
\begin{verbatim}
d(x^2,x)
\end{verbatim}}

$2x$

\bigskip
Extend the argument list for multiderivatives.

{\color{blue}
\begin{verbatim}
f = 1 / (x + y)
d(f,x,y)
\end{verbatim}}

$\displaystyle
\frac{2}{(x+y)^3}
$

{\color{blue}
\begin{verbatim}
d(sin(x),x,x)
\end{verbatim}}

$-\sin(x)$

\bigskip
Another syntax for $n$th derivative.

{\color{blue}
\begin{verbatim}
d(sin(x),x,2)
\end{verbatim}}

$-\sin(x)$

\bigskip
The gradient of $f$ is returned for vector $x$ in \verb$d(f,x)$.

{\color{blue}
\begin{verbatim}
r = sqrt(x^2 + y^2)
d(r,(x,y))
\end{verbatim}
}

$\displaystyle
\begin{bmatrix}
{\displaystyle \frac{x}{(x^2+y^2)^{1/2}}}
\\
\\
{\displaystyle \frac{y}{(x^2+y^2)^{1/2}}}
\end{bmatrix}
$

\bigskip
The $f$ in \verb$d(f,x)$ can be a vector or higher rank function.
Gradient increases rank by one.

{\color{blue}
\begin{verbatim}
F = (x^2,y^2)
X = (x,y)
d(F,X)
\end{verbatim}
}

$\displaystyle
\begin{bmatrix}
2x & 0
\\[1ex]
0 & 2y
\end{bmatrix}
$

\newpage

\section{Template functions}

Function $f$ in \verb$d(f,x)$ does not have to be defined,
it can be a template function with just a name and an argument list.
The argument list determines the result.
For example, \verb$d(f(x),x)$ evaluates to itself because $f$ depends on $x$.
However, \verb$d(f(x),y)$ evaluates to zero because $f$ does not depend on $y$.

\bigskip
Example 1. $f(x)$ depends on $x$.

{\color{blue}
\begin{verbatim}
d(f(x),x)
\end{verbatim}}

$\operatorname{d}(f(x),x)$

\bigskip
Example 2. $f(x)$ does not depend on $y$.

{\color{blue}
\begin{verbatim}
d(f(x),y)
\end{verbatim}}

$0$

\bigskip
Example 3. $f(x,y)$ depends on both $x$ and $y$.

{\color{blue}
\begin{verbatim}
d(f(x,y),y)
\end{verbatim}}

$\operatorname{d}(f(x,y),y)$

\bigskip
Example 4. $f()$ is a wildcard that matches any symbol.

{\color{blue}
\begin{verbatim}
d(f(),t)
\end{verbatim}}

$\operatorname{d}(f(),t)$

\bigskip
Template functions are useful for working with differential forms.
For example, show that
\begin{equation*}
\nabla\cdot(\nabla\times\mathbf F)=0
\end{equation*}

{\color{blue}
\begin{verbatim}
F = (Fx(),Fy(),Fz())
div(curl(F))
\end{verbatim}}

$0$

\newpage

\section{Laplacian}

The Laplacian $\nabla^2$ is the divergence of the gradient of scalar function $f$.
\begin{equation*}
\nabla^2f=\nabla\cdot\nabla f
=\frac{\partial^2 f}{\partial x^2}
+\frac{\partial^2 f}{\partial y^2}
+\frac{\partial^2 f}{\partial z^2}
\end{equation*}

{\color{blue}
\begin{verbatim}
div(grad(f()))
\end{verbatim}}

$\displaystyle
\operatorname{\rm d}(\operatorname{\rm d}(f(),x),x)
+\operatorname{\rm d}(\operatorname{\rm d}(f(),y),y)
+\operatorname{\rm d}(\operatorname{\rm d}(f(),z),z)
$

\bigskip
This is the vector Laplacian.
\begin{equation*}
\nabla^2\mathbf A
=\nabla\cdot\nabla\mathbf A-\nabla\times(\nabla\times\mathbf A)
=\begin{pmatrix}
\nabla^2A_x
\\
\nabla^2A_y
\\
\nabla^2A_z
\end{pmatrix}
=\frac{\partial^2\mathbf A}{\partial x^2}
+\frac{\partial^2\mathbf A}{\partial y^2}
+\frac{\partial^2\mathbf A}{\partial z^2}
\end{equation*}

{\color{blue}
\begin{verbatim}
A = (Ax(),Ay(),Az())
div(grad(A)) - curl(curl(A)) == d(A,x,x) + d(A,y,y) + d(A,z,z)
\end{verbatim}}

$1$

\bigskip
Show that
\begin{equation*}
\nabla\cdot\nabla\mathbf A=\nabla(\nabla\cdot\mathbf A)
\end{equation*}

{\color{blue}
\begin{verbatim}
div(grad(A)) == grad(div(A))
\end{verbatim}}

$1$

\newpage

\section{Integral}

\verb$integral(f,x)$ returns the integral of $f$ with respect to $x$.

{\color{blue}
\begin{verbatim}
integral(x^2,x)
\end{verbatim}
}

$\displaystyle \tfrac{1}{3}x^3$

\bigskip
Extend the argument list for multiple integrals.

{\color{blue}
\begin{verbatim}
f = x y
integral(f,x,y)
\end{verbatim}
}

$\displaystyle \tfrac{1}{4}x^2y^2$

\bigskip
\verb$defint(f,x,a,b)$
computes the definite integral of $f$ with respect to $x$ evaluated from
$a$ to $b$.
The argument list can be extended for multiple integrals.
The following example computes the integral of $f=x^2$
over the domain of a semicircle.
For each $x$ along the abscissa, $y$ ranges from 0 to $\sqrt{1-x^2}$.

{\color{blue}
\begin{verbatim}
defint(x^2, y, 0, sqrt(1 - x^2), x, -1, 1)
\end{verbatim}
}

$\displaystyle \tfrac{1}{8}\pi$

\bigskip
Alternatively, \verb$eval$ can be used to compute a definite integral step by step.

{\color{blue}
\begin{verbatim}
I = integral(x^2,y)
I = eval(I,y,sqrt(1 - x^2)) - eval(I,y,0)
I = integral(I,x)
eval(I,x,1) - eval(I,x,-1)
\end{verbatim}
}

$\displaystyle \tfrac{1}{8}\pi$

\bigskip
Here is a useful trick.
Integrals involving sine and cosine
can often be solved using exponentials.
For example, the definite integral
\begin{equation*}
\int_0^{2\pi}\left(\sin^4t-2\cos^3(t/2)\sin t\right)dt
\end{equation*}

can be solved as follows.

{\color{blue}
\begin{verbatim}
f = sin(t)^4 - 2 cos(t/2)^3 sin(t)
f = circexp(f)
defint(f, t, 0, 2 pi)
\end{verbatim}
}

$\displaystyle \tfrac{3}{4}\pi-\tfrac{16}{5}$

\newpage

\section{Arc length}

Let $g(t)$ be a parametric function that draws a curve in $\mathbb R^n$.
The arc length from $g(a)$ to $g(b)$ is given by
\begin{equation*}
\int_a^b|g'(t)|\,dt
\end{equation*}

where $|g'(t)|$ is the length of the tangent vector at $g(t)$.

\bigskip
Example 1.
Find the length of the curve $y=x^2$ from $x=0$ to $x=1$.

{\color{blue}
\begin{verbatim}
g = (t,t^2)
defint(abs(d(g,t)),t,0,1)
\end{verbatim}}

$
\tfrac{1}{2}\;5^{1/2}
-\tfrac{1}{4}\log(2)
+\tfrac{1}{4}\log(2\;5^{1/2}+4)
$

{\color{blue}
\begin{verbatim}
float
\end{verbatim}}

$1.47894$

\bigskip
As expected, the result is greater than $\sqrt2\approx1.414$,
the length of a straight line from $(0,0)$ to $(1,1)$.

\bigskip
The following script does a discrete computation of the arc length
by dividing the curve into 100 pieces.

{\color{blue}
\begin{verbatim}
g(t) = (t,t^2)
h(k) = abs(g(k/100.0) - g((k-1)/100.0))
sum(k,1,100,h(k))
\end{verbatim}}

$1.47894$

\bigskip
As expected, the discrete result matches the analytic result.

\bigskip
Example 2.
Find the length of the curve $y=x^{3/2}$ from the origin to
$x=\tfrac{4}{3}$.

{\color{blue}
\begin{verbatim}
g = (t,t^(3/2))
defint(abs(d(g,t)),t,0,4/3)
\end{verbatim}}

$\displaystyle \tfrac{56}{27}$

\newpage

\section{Line integral}

There are two kinds of line integrals,
one for scalar fields and one for vector fields.
The following table shows how both are based on the calculation of
arc length.

\begin{center}
\begin{tabular}{|l|l|l|}
\hline
& Abstract form
& Computable form
\\
\hline
 & &\\
Arc length
& $\displaystyle{\int_C ds}$
& $\displaystyle{\int_a^b |g'(t)|\,dt}$\\
 & &\\
\hline
 & & \\
Line integral, scalar field
& $\displaystyle{\int_C f\,ds}$
& $\displaystyle{\int_a^b f(g(t))\,|g'(t)|\,dt}$\\
& &\\
\hline
 & & \\
Line integral, vector field
& $\displaystyle{\int_C(F\cdot u)\,ds}$
& $\displaystyle{\int_a^b F(g(t))\cdot g'(t)\,dt}$\\
 & & \\
\hline
\end{tabular}
\end{center}

Note that for the measure $ds$ we have
\begin{equation*}
ds=|g'(t)|\,dt
\end{equation*}

For vector fields, symbol $u$ is the unit tangent vector
\begin{equation*}
u=\frac{g'(t)}{|g'(t)|}
\end{equation*}

Note that $u$ cancels with $ds$ as follows.
\begin{equation*}
\int_C(F\cdot u)\,ds
=\int_a^b
\left(F(g(t))\cdot\frac{g'(t)}{|g'(t)|}\right)
|g'(t)|\,dt
=\int_a^b F(g(t))\cdot g'(t)\,dt
\end{equation*}

Example 1. Evaluate
\begin{equation*}
\int_Cx\,ds\quad\hbox{and}\quad\int_Cx\,dx
\end{equation*}

where $C$ is a straight line from $(0,0)$ to $(1,1)$.

\bigskip
Although the integrals appear similar,
the first is over a scalar field and the second is over a vector field.

\bigskip
For $\int_Cx\,ds$ we have

{\color{blue}
\begin{verbatim}
x = t
y = t
g = (x,y)
defint(x abs(d(g,t)), t, 0, 1)
\end{verbatim}}

$\displaystyle \frac{1}{2^{1/2}}$

\bigskip
For $\int_Cx\,dx$ we have

{\color{blue}
\begin{verbatim}
x = t
y = t
g = (x,y)
F = (x,0)
defint(dot(F,d(g,t)), t, 0, 1)
\end{verbatim}}

$\displaystyle \tfrac{1}{2}$

\bigskip
The following line integral problems are from
{\it Advanced Calculus, Fifth Edition} by Wilfred Kaplan.

\bigskip
Example 2. Evaluate $\int y^2\,dx$ along the straight
line from $(0,0)$ to $(2,2)$.

\bigskip
The following solution parametrizes $x$ and $y$ so that
the endpoint $(2,2)$ corresponds to $t=1$.

{\color{blue}
\begin{verbatim}
x = 2 t
y = 2 t
g = (x,y)
F = (y^2,0)
defint(dot(F,d(g,t)), t, 0, 1)
\end{verbatim}}

$\displaystyle \tfrac{8}{3}$

\bigskip
Example 3. Evaluate $\int z\,dx+x\,dy+y\,dz$
along the path
$x=2t+1$, $y=t^2$, $z=1+t^3$, $0\le t\le 1$.

{\color{blue}
\begin{verbatim}
x = 2 t + 1
y = t^2
z = 1 + t^3
g = (x,y,z)
F = (z,x,y)
defint(dot(F,d(g,t)), t, 0, 1)
\end{verbatim}}

$\displaystyle \tfrac{163}{30}$

\newpage

\section{Surface area}

Let $S$ be a surface parameterized by $x$ and $y$.
That is, let $S=(x,y,z)$ where $z=f(x,y)$.
The tangent lines at a point on $S$ form a tiny parallelogram.
The area $a$ of the parallelogram is given by the magnitude of the cross product.
\begin{equation*}
a=\left|\frac{\partial S}{\partial x}\times\frac{\partial S}{\partial y}\right|
\end{equation*}

By summing over all the parallelograms we obtain the total surface area $A$.
Hence
\begin{equation*}
A=\int\int dA=\int\int a\,dx\,dy
\end{equation*}

The following example computes the surface area of a unit disk
parallel to the $xy$ plane.

{\color{blue}
\begin{verbatim}
z = 2
S = (x,y,z)
a = abs(cross(d(S,x),d(S,y)))
defint(a,y,-sqrt(1 - x^2),sqrt(1 - x^2),x,-1,1)
\end{verbatim}
}

$\displaystyle \pi$

\bigskip
The result is $\pi$, the area of a unit circle, which is what we expect.
The following example computes the surface area of $z=x^2+2y$ over
a unit square.

{\color{blue}
\begin{verbatim}
z = x^2 + 2y
S = (x,y,z)
a = abs(cross(d(S,x),d(S,y)))
defint(a,x,0,1,y,0,1)
\end{verbatim}
}

$\displaystyle \tfrac{5}{8}\log(5)+\tfrac{3}{2}$

\bigskip
The following exercise is from
{\it Multivariable Mathematics} by Williamson and Trotter, p. 598.
Find the area of the spiral ramp defined by
\begin{equation*}
S=\begin{pmatrix}u\cos v\\\ u\sin v\\ v\end{pmatrix},\quad 0\le u\le1,\quad 0\le v\le3\pi
\end{equation*}

{\color{blue}
\begin{verbatim}
x = u cos(v)
y = u sin(v)
z = v
S = (x,y,z)
a = circexp(abs(cross(d(S,u),d(S,v))))
defint(a,u,0,1,v,0,3pi)
\end{verbatim}
}

$\displaystyle \frac{3\pi}{2^{1/2}}+\tfrac{3}{2}\pi\log\left(2^{1/2}+1\right)$

{\color{blue}
\begin{verbatim}
float
\end{verbatim}
}

$\displaystyle 10.8177$

\newpage

\section{Surface integral}

A surface integral is like adding up all the wind on a sail.
In other words, we want to compute
$$\int\!\!\!\int{\bf F\cdot n}\,dA$$
where ${\bf F\cdot n}$ is the amount of wind normal to a tiny parallelogram $dA$.
The integral sums over the entire area of the sail.
Let $S$ be the surface of the sail parameterized by $x$ and $y$.
(In this model, the $z$ direction points downwind.)
By the properties of the cross product we have the following for the unit normal $\bf n$
and for $dA$.
$${\bf n}=\frac{{\frac{\partial S}{\partial x}\times\frac{\partial S}{\partial y}}}
{{\left|\frac{\partial S}{\partial x}\times\frac{\partial S}{\partial y}\right|}}\qquad
dA=\left|\frac{\partial S}{\partial x}\times\frac{\partial S}{\partial y}\right|\,dx\,dy$$
Hence
$$\int\!\!\!\int{\bf F\cdot n}\,dA=\int\!\!\!\int{\bf F}\cdot
\left({\frac{\partial S}{\partial x}\times\frac{\partial S}{\partial y}}\right)\,dx\,dy$$

\bigskip
The following exercise is from
{\it Advanced Calculus} by Wilfred Kaplan, p.~313.
Evaluate the surface integral
$$\int\!\!\!\int_S{\bf F\cdot n}\,d\sigma$$

where ${\bf F}=xy^2z{\bf i}-2x^3{\bf j}+yz^2{\bf k}$, $S$ is the surface
$z=1-x^2-y^2$, $x^2+y^2\le1$ and $\bf n$ is upper.

\bigskip
Note that the surface intersects the $xy$ plane in a circle.
By the right hand rule, crossing $x$ into $y$ yields $\bf n$ pointing upwards hence
$${\bf n}\,d\sigma=\left({\frac{\partial S}{\partial x}\times\frac{\partial S}{\partial y}}\right)\,dx\,dy$$

The following code computes the surface integral.
The symbols $f$ and $h$ are used as temporary variables.

{\color{blue}
\begin{verbatim}
z = 1 - x^2 - y^2
F = (x y^2 z, -2 x^3, y z^2)
S = (x,y,z)
f = dot(F,cross(d(S,x),d(S,y)))
h = sqrt(1 - x^2)
defint(f, y, -h, h, x, -1, 1)
\end{verbatim}
}

$\displaystyle \tfrac{1}{48}\pi$

\newpage

\section{Green's theorem}

This is Green's theorem.
\begin{equation*}
\oint\left(P\,dx+Q\,dy\right)
=\int\!\!\!\int
\left(\frac{\partial Q}{\partial x}-\frac{\partial P}{\partial y}\right)
\,dx\,dy
\end{equation*}

In words, a line integral and a surface integral can yield
the same result.

\bigskip
Example 1.
The following exercise is from {\it Advanced Calculus}
by Wilfred Kaplan, p.~287.
Evaluate $\oint (2x^3-y^3)\,dx+(x^3+y^3)\,dy$ around the circle
$x^2+y^2=1$ using Green's theorem.

\bigskip
Use polar coordinates to solve the surface integral.

{\color{blue}
\begin{verbatim}
P = 2x^3 - y^3
Q = x^3 + y^3
f = d(Q,x) - d(P,y)
x = r cos(theta)
y = r sin(theta)
defint(f r, r, 0, 1, theta, 0, 2 pi)
\end{verbatim}
}

$\displaystyle \tfrac{3}{2}\pi$

\bigskip
The \verb$defint$ integrand is $f\,r$ because $r\,dr\,d\theta=dx\,dy$.

\bigskip
Now let us try computing the line integral side of Green's theorem
and see if we get the same result.
We need to use the trick of converting sine and cosine to exponentials
so that Eigenmath can find the solution.

{\color{blue}
\begin{verbatim}
x = cos(t)
y = sin(t)
P = 2x^3 - y^3
Q = x^3 + y^3
f = P d(x,t) + Q d(y,t)
f = circexp(f)
defint(f, t, 0, 2 pi)
\end{verbatim}
}

$\displaystyle \tfrac{3}{2}\pi$

\bigskip
Example 2.
Compute both sides of Green's theorem for
$F=(1-y,x)$ over the disk $x^2+y^2\le4$.

\bigskip
First compute the line integral along the boundary of the disk.
Note that the radius of the disk is 2.

{\color{blue}
\begin{verbatim}
-- Line integral
P = 1 - y
Q = x
x = 2 cos(t)
y = 2 sin(t)
defint(P d(x,t) + Q d(y,t),t,0,2pi)
\end{verbatim}
}

$\displaystyle 8\pi$

{\color{blue}
\begin{verbatim}
-- Surface integral
x = quote(x) -- clear x
y = quote(y) -- clear y
h = sqrt(4 - x^2)
defint(d(Q,x) - d(P,y), y, -h, h, x, -2, 2)
\end{verbatim}
}

$\displaystyle 8\pi$

{\color{blue}
\begin{verbatim}
-- use polar coordinates
P = 1 - y
Q = x
f = d(Q,x) - d(P,y) -- do before change of coordinates
x = r cos(theta)
y = r sin(theta)
defint(f r, r, 0, 2, theta, 0, 2 pi)
\end{verbatim}
}

$\displaystyle 8\pi$

{\color{blue}
\begin{verbatim}
defint(f r, theta, 0, 2 pi, r, 0, 2) -- integrate over theta first
\end{verbatim}
}

$\displaystyle 8\pi$

\bigskip
In this case, Eigenmath solved both forms of the polar integral.
However, in cases where Eigenmath fails to solve a double integral, try
changing the order of integration.

\newpage

\section{Stokes's theorem}

Stokes's theorem equates a surface integral of the curl of a function
with a line integral of the same function.
In rectangular coordinates the equivalence is
%
$$
\iint_S(\mathop{\rm curl}{\bf F})\cdot{\bf n}\,d\sigma
=\oint (P\,dx+Q\,dy+R\,dz)
$$
%
where ${\bf F}=(P,Q,R)$.
For $S$ parametrized by $x$ and $y$ we have
$${\bf n}\,d\sigma=\left(
\frac{\partial S}{\partial x}\times\frac{\partial S}{\partial y}
\right)dx\,dy$$

For example,
let ${\bf F}=(y,z,x)$ and let $S$ be the part of the paraboloid
$z=4-x^2-y^2$
that is above the $xy$ plane.
The perimeter of the paraboloid is the circle $x^2+y^2=2$.
The following script computes both the surface and line integrals.
Polar coordinates are used for the line integral so that \verb$defint$ can succeed.

{\color{blue}
\begin{verbatim}
"Surface integral"
z = 4 - x^2 - y^2
F = (y,z,x)
S = (x,y,z)
z = quote(z) -- clear z for use by curl
f = dot(curl(F),cross(d(S,x),d(S,y)))
x = r cos(theta)
y = r sin(theta)
defint(f r, r, 0, 2, theta, 0, 2 pi)

"Line integral"
x = 2 cos(t)
y = 2 sin(t)
z = 4 - x^2 - y^2
P = y
Q = z
R = x
f = P d(x,t) + Q d(y,t) + R d(z,t)
f = circexp(f)
defint(f, t, 0, 2 pi)
\end{verbatim}
}

This is the result when the script runs.
Both the surface integral and the line integral
yield the same result.

\bigskip
Surface integral

$\displaystyle -4\pi$

Line integral

$\displaystyle -4\pi$

\newpage

\section{Feature index}

\subsection*{abs($x$)}

Returns the absolute value or vector length of $x$.

{\color{blue}
\begin{verbatim}
X = (x,y,z)
abs(X)
\end{verbatim}
}

\noindent
$\left(x^2+y^2+z^2\right)^{1/2}$

\subsection*{adj($m$)}

Returns the adjunct of matrix $m$.
Adjunct is equal to determinant times inverse.

{\color{blue}
\begin{verbatim}
A = ((a,b),(c,d))
adj(A) == det(A) inv(A)
\end{verbatim}
}

\noindent
$1$

\subsection*{and($a,b,\ldots$)}

Returns 1 if all arguments are true (nonzero).
Returns 0 otherwise.

{\color{blue}
\begin{verbatim}
and(1=1,2=2)
\end{verbatim}
}

\noindent
$1$

\subsection*{arccos($x$)}

Returns the arc cosine of $x$.

{\color{blue}
\begin{verbatim}
arccos(1/2)
\end{verbatim}
}

\noindent
$\tfrac{1}{3}\pi$

\subsection*{arccosh($x$)}

Returns the arc hyperbolic cosine of $x$.

\subsection*{arcsin($x$)}

Returns the arc sine of $x$.

{\color{blue}
\begin{verbatim}
arcsin(1/2)
\end{verbatim}
}

\noindent
$\tfrac{1}{6}\pi$

\subsection*{arcsinh($x$)}

Returns the arc hyperbolic sine of $x$.

\subsection*{arctan($y,x$)}

Returns the arc tangent of $y$ over $x$.
If $x$ is omitted then $x=1$ is used.

{\color{blue}
\begin{verbatim}
arctan(1,0)
\end{verbatim}
}

\noindent
$\tfrac{1}{2}\pi$

\subsection*{arctanh($x$)}

Returns the arc hyperbolic tangent of $x$.

\subsection*{arg($z$)}

Returns the angle of complex $z$.

{\color{blue}
\begin{verbatim}
arg(2 - 3i)
\end{verbatim}
}

\noindent
$\arctan(-3,2)$

\subsection*{binding($s$)}

The result of evaluating a symbol can differ from the symbol's binding.
For example, the result may be expanded.
The {\tt binding} function returns the actual binding of a symbol.

{\color{blue}
\begin{verbatim}
p = quote((x + 1)^2)
p
\end{verbatim}
}

\noindent
$p=x^2+2x+1$

{\color{blue}
\begin{verbatim}
binding(p)
\end{verbatim}
}

\noindent
$(x+1)^2$

\subsection*{ceiling($x$)}

Returns the smallest integer greater than or equal to $x$.

{\color{blue}
\begin{verbatim}
ceiling(1/2)
\end{verbatim}
}

\noindent
$1$

\subsection*{check($x$)}

If $x$ is true (nonzero) then continue, else stop.
Expression $x$ can include the relational operators
\verb$=$,
\verb$==$,
\verb$<$,
\verb$<=$,
\verb$>$,
\verb$>=$.
Use the
\verb$not$
function to test for inequality.

{\color{blue}
\begin{verbatim}
A = exp(i pi)
B = -1
check(A == B) -- stop here if A not equal to B
\end{verbatim}
}

\subsection*{choose($n,k$)}

Returns the binomial coefficient $n$ choose $k$.

{\color{blue}
\begin{verbatim}
choose(52,5) -- number of poker hands
\end{verbatim}
}

\noindent
$2598960$

\subsection*{circexp($x$)}

Returns expression $x$ with circular and hyperbolic functions
converted to exponentials.

{\color{blue}
\begin{verbatim}
circexp(cos(x) + i sin(x))
\end{verbatim}
}

\noindent
$\exp(ix)$

\subsection*{clear}

Clears all symbol definitions.

\subsection*{clock($z$)}

Returns complex $z$ in polar form with base of negative 1 instead of $e$.

{\color{blue}
\begin{verbatim}
clock(2 - 3i)
\end{verbatim}
}

\noindent
$13^{1/2}\,(-1)^{\arctan(-3,2)/\pi}$

\subsection*{cofactor($m,i,j$)}

Returns the cofactor of matrix $m$ for row $i$ and column $j$.

{\color{blue}
\begin{verbatim}
A = ((a,b),(c,d))
cofactor(A,1,2) == adj(A)[2,1]
\end{verbatim}
}

\noindent
1

\subsection*{conj($z$)}

Returns the complex conjugate of $z$.

{\color{blue}
\begin{verbatim}
conj(2 - 3i)
\end{verbatim}
}

\noindent
$2 + 3 i$

\subsection*{contract($a,i,j$)}

Returns tensor $a$ summed over indices $i$ and $j$.
If $i$ and $j$ are omitted then 1 and 2 are used.
The expression {\tt contract(m)} computes the trace of matrix $m$.

{\color{blue}
\begin{verbatim}
A = ((a,b),(c,d))
contract(A)
\end{verbatim}
}

\noindent
$a + d$

\subsection*{cos($x$)}

Returns the cosine of $x$.

{\color{blue}
\begin{verbatim}
cos(pi/4)
\end{verbatim}
}

\noindent
$\displaystyle \frac{1}{2^{1/2}}$

\subsection*{cosh($x$)}

Returns the hyperbolic cosine of $x$.

{\color{blue}
\begin{verbatim}
circexp(cosh(x))
\end{verbatim}
}

\noindent
$\tfrac{1}{2}\exp(-x)+\tfrac{1}{2}\exp(x)$

\subsection*{cross($u,v$)}

Returns the cross product of vectors $u$ and $v$.

\subsection*{curl($v$)}

Returns the curl of vector $v$ with respect to symbols \verb$x$, \verb$y$, and \verb$z$.

\subsection*{d($f,x,\ldots$)}

Returns the partial derivative of $f$ with respect to $x$ and any additional arguments.

{\color{blue}
\begin{verbatim}
d(sin(x),x)
\end{verbatim}
}

\noindent
$\cos(x)$

\bigskip
\noindent
Multiderivatives are computed by extending the argument list.

{\color{blue}
\begin{verbatim}
d(sin(x),x,x)
\end{verbatim}
}

\noindent
$-\sin(x)$

\bigskip
\noindent
A numeric argument $n$ computes the $n$th derivative with respect to the previous symbol.

{\color{blue}
\begin{verbatim}
d(sin(x y),x,2,y,2)
\end{verbatim}
}

\noindent
$x^2y^2\sin(xy)-4xy\cos(xy)-2\sin(xy)$

\bigskip
\noindent
Argument $f$ can be a tensor of any rank.
Argument $x$ can be a vector.
When $x$ is a vector the result is the gradient of $f$.

{\color{blue}
\begin{verbatim}
F = (f(),g(),h())
X = (x,y,z)
d(F,X)
\end{verbatim}
}

\noindent
$\displaystyle
\begin{bmatrix}
\operatorname{d}(f(),x) & \operatorname{d}(f(),y) &  \operatorname{d}(f(),z)
\\[1ex]
\operatorname{d}(g(),x) & \operatorname{d}(g(),y) &  \operatorname{d}(g(),z)
\\[1ex]
\operatorname{d}(h(),x) & \operatorname{d}(h(),y) &  \operatorname{d}(h(),z)
\end{bmatrix}
$

\bigskip
\noindent
Symbol \verb$d$ can be used as a variable name.
Doing so does not conflict with function \verb$d$.

\bigskip
\noindent
Symbol \verb$d$ can be redefined as a different function.
The function \verb$derivative$, a synonym for \verb$d$, can be used to obtain a partial derivative.

\subsection*{defint($f,x,a,b$)}

Returns the definite integral of $f$ with respect to $x$
evaluated from $a$ to $b$.
The argument list can be extended for multiple integrals
as shown in the following example.

{\color{blue}
\begin{verbatim}
f = (1 + cos(theta)^2) sin(theta)
-- integrate over theta then over phi
defint(f, theta, 0, pi, phi, 0, 2 pi)
\end{verbatim}
}

\noindent
$\tfrac{16}{3}\pi$

\subsection*{denominator($x$)}

Returns the denominator of expression $x$.

{\color{blue}
\begin{verbatim}
denominator(a/b)
\end{verbatim}
}

\noindent
$b$

\subsection*{det($m$)}

Returns the determinant of matrix $m$.

{\color{blue}
\begin{verbatim}
A = ((a,b),(c,d))
det(A)
\end{verbatim}
}

\noindent
$a d - b c$

\subsection*{dim($a,n$)}

Returns the dimension of the $n$th index of tensor $a$.
Index numbering starts with 1.

{\color{blue}
\begin{verbatim}
A = ((1,2),(3,4),(5,6))
dim(A,1)
\end{verbatim}
}

\noindent
$3$

\subsection*{div($v$)}

Returns the divergence of vector $v$ with respect to symbols \verb$x$, \verb$y$, and \verb$z$.

\subsection*{do($a,b,\ldots$)}

Evaluates each argument from left to right.
Returns the result of the final argument.

{\color{blue}
\begin{verbatim}
do(A=1,B=2,A+B)
\end{verbatim}
}

\noindent
$3$

\subsection*{dot($a,b,\ldots$)}

Returns the dot product of vectors, matrices, and tensors.
Also known as the matrix product.
Arguments are evaluated from right to left.
The following example solves for $X$ in $AX=B$.

{\color{blue}
\begin{verbatim}
A = ((1,2),(3,4))
B = (5,6)
X = dot(inv(A),B)
X
\end{verbatim}
}

\noindent
$\displaystyle
\begin{bmatrix}
-4
\\[1ex]
\tfrac{9}{2}
\end{bmatrix}
$

\subsection*{eigenvec($m$)}

Returns eigenvectors for matrix $m$.
Matrix $m$ is required to be numerical, real, and symmetric.
The return value is a matrix with each column an eigenvector.
Eigenvalues are obtained as shown.

{\color{blue}
\begin{verbatim}
A = ((1,2,3),(2,6,4),(3,4,5))
Q = eigenvec(A)
D = dot(transpose(Q),A,Q) -- eigenvalues on the diagonal of D
dot(Q,D,transpose(Q))
\end{verbatim}
}

\noindent
$\displaystyle
\begin{bmatrix}
1 & 2 & 3
\\[1ex]
2 & 6 & 4
\\[1ex]
3 & 4 & 5
\end{bmatrix}
$

\subsection*{eval($f,x,a$)}

Returns expression $f$ evaluated at $x$ equals $a$.
The argument list can be extended as shown.

{\color{blue}
\begin{verbatim}
f = sqrt(x^2 + y^2)
eval(f,x,3,y,4) -- evaluate f at x=3 and y=4
\end{verbatim}
}

\noindent
5

\bigskip
\noindent
The \verb$eval$ function can be used for substitution.
In the following example, \verb$eval$ is used to replace
\verb$x$ with \verb$cos(theta)$.

{\color{blue}
\begin{verbatim}
-- associated legendre of cos theta

P(l,m) = test(m < 0, (-1)^m (l + m)! / (l - m)! P(l,-m),
         1 / (2^l l!) sin(theta)^m *
         eval(d((x^2 - 1)^l,x,l + m),x,cos(theta)))

P(2,-1)
\end{verbatim}
}

$\displaystyle
-\tfrac{1}{2}\cos(\theta)\sin(\theta)
$

\subsection*{exp($x$)}

Returns the exponential of $x$.

{\color{blue}
\begin{verbatim}
exp(i pi)
\end{verbatim}
}

\noindent
$-1$

\subsection*{expcos($z$)}

Returns the cosine of $z$ in exponential form.

{\color{blue}
\begin{verbatim}
expcos(z)
\end{verbatim}
}

\noindent
$\displaystyle \tfrac{1}{2}\exp(iz)+\tfrac{1}{2}\exp(-iz)$

\subsection*{expcosh($z$)}

Returns the hyperbolic cosine of $z$ in exponential form.

{\color{blue}
\begin{verbatim}
expcosh(z)
\end{verbatim}
}

\noindent
$\displaystyle \tfrac{1}{2}\exp(-z)+\tfrac{1}{2}\exp(z)$

\subsection*{expsin($z$)}

Returns the sine of $z$ in exponential form.

{\color{blue}
\begin{verbatim}
expsin(z)
\end{verbatim}
}

\noindent
$\displaystyle -\tfrac{1}{2}i\exp(iz)+\tfrac{1}{2}i\exp(-iz)$

\subsection*{expsinh($z$)}

Returns the hyperbolic sine of $z$ in exponential form.

{\color{blue}
\begin{verbatim}
expsinh(z)
\end{verbatim}
}

\noindent
$\displaystyle -\tfrac{1}{2}\exp(-z)+\tfrac{1}{2}\exp(z)$

\subsection*{exptan($z$)}

Returns the tangent of $z$ in exponential form.

{\color{blue}
\begin{verbatim}
exptan(z)
\end{verbatim}
}

\noindent
$\displaystyle \frac{i}{\exp(2iz)+1}-\frac{i\exp(2iz)}{\exp(2iz)+1}$

\subsection*{exptanh($z$)}

Returns the hyperbolic tangent of $z$ in exponential form.

{\color{blue}
\begin{verbatim}
exptanh(z)
\end{verbatim}
}

\noindent
$\displaystyle -\frac{1}{\exp(2z)+1}+\frac{\exp(2z)}{\exp(2z)+1}$

\subsection*{factorial($n$)}

Returns the factorial of $n$.
The expression {\tt n!} can also be used.

{\color{blue}
\begin{verbatim}
20!
\end{verbatim}
}

\noindent
$2432902008176640000$

\subsection*{float($x$)}

Returns expression $x$ with rational numbers and integers converted to
floating point values.
The symbol {\tt pi} and the natural number are also converted.

{\color{blue}
\begin{verbatim}
float(212^17)
\end{verbatim}
}

\noindent
$\displaystyle 3.52947\times 10^{39}$

\subsection*{floor($x$)}

Returns the largest integer less than or equal to $x$.

{\color{blue}
\begin{verbatim}
floor(1/2)
\end{verbatim}
}

\noindent
$0$

\subsection*{for($i,j,k,a,b,\ldots$)}

For $i$ equals $j$ through $k$ evaluate $a$, $b$, etc.

{\color{blue}
\begin{verbatim}
for(k,1,3,A=k,print(A))
\end{verbatim}
}

\noindent
$A=1$\\
$A=2$\\
$A=3$

\bigskip
\noindent
Note: The original value of $i$ is restored after {\tt for} completes.
If symbol {\tt i} is used for index variable $i$
then the imaginary unit is overridden in the scope of {\tt for}.

\subsection*{grad($f$)}

Returns the gradient \verb$d(f,(x,y,z))$.

{\color{blue}
\begin{verbatim}
grad(f())
\end{verbatim}
}

\noindent
$\displaystyle
\begin{bmatrix}
{\rm d}(f(),x)
\\[1ex]
{\rm d}(f(),y)
\\[1ex]
{\rm d}(f(),z)
\end{bmatrix}
$

\subsection*{hadamard($a,b,\ldots$)}

Returns the Hadamard (element-wise) product.

{\color{blue}
\begin{verbatim}
X = (a,b,c)
hadamard(X,X)
\end{verbatim}
}

\noindent
$\displaystyle
\begin{bmatrix}
a^2
\\[1ex]
b^2
\\[1ex]
c^2
\end{bmatrix}
$

\subsection*{i}

Symbol {\tt i} is initialized to the imaginary unit $\sqrt{-1}$.

{\color{blue}
\begin{verbatim}
exp(i pi)
\end{verbatim}
}

\noindent
$-1$

\bigskip
\noindent
Note: It is ok to clear or redefine {\tt i} and use the symbol for something else.

\subsection*{imag($z$)}

Returns the imaginary part of complex $z$.

{\color{blue}
\begin{verbatim}
imag(2 - 3i)
\end{verbatim}
}

\noindent
$-3$

\subsection*{infixform($x$)}

Converts expression $x$ to a string and returns the result.

{\color{blue}
\begin{verbatim}
p = (x + 1)^2
infixform(p)
\end{verbatim}
}

\noindent
\verb$x^2 + 2 x + 1$

\subsection*{inner($a,b,\ldots$)}

Returns the inner product of vectors, matrices, and tensors.
Also known as the matrix product.

{\color{blue}
\begin{verbatim}
A = ((a,b),(c,d))
B = (x,y)
inner(A,B)
\end{verbatim}
}

\noindent
$\displaystyle
\begin{bmatrix}
ax+by
\\[1ex]
cx+dy
\end{bmatrix}
$

\bigskip
\noindent
Note: {\tt inner} and {\tt dot} are the same function.

\subsection*{integral($f,x$)}

Returns the integral of $f$ with respect to $x$.

{\color{blue}
\begin{verbatim}
integral(x^2,x)
\end{verbatim}
}

\noindent
$\displaystyle \tfrac{1}{3}x^3$

\subsection*{inv($m$)}

Returns the inverse of matrix $m$.

{\color{blue}
\begin{verbatim}
A = ((1,2),(3,4))
inv(A)
\end{verbatim}
}

\noindent
$\displaystyle
\begin{bmatrix}
-2 & 1
\\[1ex]
\tfrac{3}{2} & -\tfrac{1}{2}
\end{bmatrix}
$

\subsection*{j}

Set {\tt j=sqrt(-1)} to use {\tt j} for the imaginary unit instead of {\tt i}.

{\color{blue}
\begin{verbatim}
j = sqrt(-1)
1/sqrt(-1)
\end{verbatim}
}

\noindent
$-j$

\subsection*{kronecker($a,b,\ldots$)}

Returns the Kronecker product of vectors and matrices.

{\color{blue}
\begin{verbatim}
A = ((1,2),(3,4))
B = ((a,b),(c,d))
kronecker(A,B)
\end{verbatim}
}

\noindent
$\displaystyle
\begin{bmatrix}
a & b & 2a & 2b
\\[1ex]
c & d & 2c & 2d
\\[1ex]
3a & 3b & 4a & 4b
\\[1ex]
3c & 3d & 3c & 4d
\end{bmatrix}
$

\subsection*{last}

The result of the previous calculation is stored in {\tt last}.

{\color{blue}
\begin{verbatim}
212^17
\end{verbatim}
}

\noindent
$3529471145760275132301897342055866171392$

{\color{blue}
\begin{verbatim}
last^(1/17)
\end{verbatim}
}

\noindent
$212$

\bigskip
\noindent
Symbol \verb$last$ is an implied argument when a function has no argument list.

{\color{blue}
\begin{verbatim}
212^17
\end{verbatim}
}

\noindent
$3529471145760275132301897342055866171392$

{\color{blue}
\begin{verbatim}
float
\end{verbatim}
}

\noindent
$\displaystyle 3.52947\times10^{39}$

\subsection*{log($x$)}

Returns the natural logarithm of $x$.

{\color{blue}
\begin{verbatim}
log(x^y)
\end{verbatim}
}

\noindent
$y\log(x)$

\subsection*{mag($z$)}

Returns the magnitude of complex $z$.
Function {\tt mag} treats undefined symbols as real while {\tt abs} does not.

{\color{blue}
\begin{verbatim}
mag(x + i y)
\end{verbatim}
}

\noindent
$\displaystyle (x^2+y^2)^{1/2}$

\subsection*{minor($m,i,j$)}

Returns the minor of matrix $m$ for row $i$ and column $j$.

{\color{blue}
\begin{verbatim}
A = ((1,2,3),(4,5,6),(7,8,9))
minor(A,1,1) == det(minormatrix(A,1,1))
\end{verbatim}
}

\noindent
1

\subsection*{minormatrix($m,i,j$)}

Returns a copy of matrix $m$ with row $i$ and column $j$ removed.

{\color{blue}
\begin{verbatim}
A = ((1,2,3),(4,5,6),(7,8,9))
minormatrix(A,1,1)
\end{verbatim}
}

\noindent
$\displaystyle
\begin{bmatrix}
5 & 6
\\[1ex]
8 & 9
\end{bmatrix}
$

\subsection*{noexpand($x$)}

Evaluates expression $x$ without expanding products of sums.

{\color{blue}
\begin{verbatim}
noexpand((x + 1)^2 / (x + 1))
\end{verbatim}
}

\noindent
$x + 1$

\subsection*{not($x$)}

Returns 0 if $x$ is true (nonzero).
Returns 1 otherwise.

{\color{blue}
\begin{verbatim}
not(1=1)
\end{verbatim}
}

\noindent
$0$

\subsection*{nroots($p,x$)}

Returns the approximate roots of polynomials with real or complex coefficients.
Multiple roots are returned as a vector.

{\color{blue}
\begin{verbatim}
p = x^5 - 1
nroots(p,x)
\end{verbatim}
}

\noindent
$\displaystyle
\begin{bmatrix}
1
\\[1ex]
-0.809017 + 0.587785\,i
\\[1ex]
-0.809017 - 0.587785\,i
\\[1ex]
0.309017 + 0.951057\,i
\\[1ex]
0.309017 - 0.951057\,i
\end{bmatrix}
$

\subsection*{numerator($x$)}

Returns the numerator of expression $x$.

{\color{blue}
\begin{verbatim}
numerator(a/b)
\end{verbatim}
}

\noindent
$a$

\subsection*{or($a,b,\ldots$)}

Returns 1 if at least one argument is true (nonzero).
Returns 0 otherwise.

{\color{blue}
\begin{verbatim}
or(1=1,2=2)
\end{verbatim}
}

\noindent
$1$

\subsection*{outer($a,b,\ldots$)}

Returns the outer product of vectors, matrices, and tensors.

{\color{blue}
\begin{verbatim}
A = (a,b,c)
B = (x,y,z)
outer(A,B)
\end{verbatim}
}

\noindent
$\displaystyle
\begin{bmatrix}
a x & a y & a z
\\[1ex]
b x & b y & b z
\\[1ex]
c x & c y & c z
\end{bmatrix}
$

\subsection*{pi}

Symbol for $\pi$.

{\color{blue}
\begin{verbatim}
exp(i pi)
\end{verbatim}
}

\noindent
$-1$

\subsection*{polar($z$)}

Returns complex $z$ in polar form.

{\color{blue}
\begin{verbatim}
polar(x - i y)
\end{verbatim}
}

\noindent
$\displaystyle (x^2+y^2)^{1/2}\exp(i\arctan(-y,x))$

\subsection*{power}

Use \verb$^$ to raise something to a power.
Use parentheses for negative powers.

{\color{blue}
\begin{verbatim}
x^(-2)
\end{verbatim}
}

\noindent
$\displaystyle \frac{1}{x^2}$

\subsection*{print($a,b,\ldots$)}

Evaluate expressions and print the results.
Useful for printing from inside a {\tt for} loop.

{\color{blue}
\begin{verbatim}
for(j,1,3,print(j))
\end{verbatim}
}

\noindent
$j=1$\newline
$j=2$\newline
$j=3$

\subsection*{product($i,j,k,f$)}

For $i$ equals $j$ through $k$ evaluate $f$.
Returns the product of all $f$.

{\color{blue}
\begin{verbatim}
product(j,1,3,x + j)
\end{verbatim}
}

\noindent
$\displaystyle x^3+6x^2+11x+6$

\bigskip
\noindent
The original value of $i$ is restored after {\tt product} completes.
If symbol {\tt i} is used for index variable $i$
then the imaginary unit is overridden in the scope of {\tt product}.

\subsection*{product($y$)}

Returns the product of components of $y$.

{\color{blue}
\begin{verbatim}
y = (1,2,3,4)
product(y)
\end{verbatim}
}

\noindent
24

\subsection*{quote($x$)}

Returns expression $x$ without evaluating it first.

{\color{blue}
\begin{verbatim}
quote((x + 1)^2)
\end{verbatim}
}

\noindent
$\displaystyle (x+1)^2$

\subsection*{rank($a$)}

Returns the number of indices that tensor $a$ has.

{\color{blue}
\begin{verbatim}
A = ((a,b),(c,d))
rank(A)
\end{verbatim}
}

\noindent
2

\subsection*{rationalize($x$)}

Returns expression $x$ with everything over a common denominator.

{\color{blue}
\begin{verbatim}
rationalize(1/a + 1/b + 1/2)
\end{verbatim}
}

\noindent
$\displaystyle \frac{2a+ab+2b}{2ab}$

\bigskip
\noindent
Note:
\verb$rationalize$
returns an unexpanded expression.
If the result is assigned to a symbol, evaluating the symbol will expand the result.
Use
\verb$binding$
to retrieve the unexpanded expression.

{\color{blue}
\begin{verbatim}
f = rationalize(1/a + 1/b + 1/2)
binding(f)
\end{verbatim}
}

\noindent
$\displaystyle \frac{2a+ab+2b}{2ab}$

\subsection*{real($z$)}

Returns the real part of complex $z$.

{\color{blue}
\begin{verbatim}
real(2 - 3i)
\end{verbatim}
}

\noindent
2

\subsection*{rect($z$)}

Returns complex $z$ in rectangular form.

{\color{blue}
\begin{verbatim}
rect(exp(i x))
\end{verbatim}
}

\noindent
$\displaystyle \cos(x)+i\sin(x)$

\subsection*{roots($p,x$)}

Returns the rational roots of a polynomial.
Multiple roots are returned as a vector.

{\color{blue}
\begin{verbatim}
p = (x + 1) (x - 2)
roots(p,x)
\end{verbatim}
}

\noindent
$\displaystyle
\begin{bmatrix}
-1
\\[1ex]
2
\end{bmatrix}
$

\bigskip
\noindent
If no roots are found then \verb$nil$ is returned.
A \verb$nil$ result is not printed so the following example uses
\verb$infixform$ to print \verb$nil$ as a string.

{\color{blue}
\begin{verbatim}
p = x^2 + 1
infixform(roots(p,x))
\end{verbatim}
}

\noindent
nil

\subsection*{rotate($u,s,k,\ldots$)}
Rotates vector $u$ and returns the result.
Vector $u$ is required to have $2^n$ elements where
$n$ is an integer from 1 to 15.
Arguments $s,k,\ldots$ are a sequence of rotation codes
where $s$ is an upper case letter and $k$ is a qubit number
from 0 to $n-1$.
Rotations are evaluated from left to right.
See the section on quantum computing for a list of rotation codes.

{\color{blue}
\begin{verbatim}
psi = (1,0,0,0)
rotate(psi,H,0)
\end{verbatim}
}

\noindent
$\displaystyle
\begin{bmatrix}
{\displaystyle \frac{1}{2^{1/2}}}
\\
\\
{\displaystyle \frac{1}{2^{1/2}}}
\\
\\
0
\\
\\
0
\end{bmatrix}
$

\subsection*{run($x$)}

Run script $x$ where $x$ evaluates to a filename string.
Useful for importing function libraries.

{\color{blue}
\begin{verbatim}
run("EVA2.txt")
\end{verbatim}
}

\noindent
For Eigenmath installed from the Mac App Store,
run files need to be put in the directory
\verb$~/Library/Containers/eigenmath/Data/$

\subsection*{simplify($x$)}

Returns expression $x$ in a simpler form.

{\color{blue}
\begin{verbatim}
simplify(sin(x)^2 + cos(x)^2)
\end{verbatim}
}

\noindent
1

\subsection*{sin($x$)}

Returns the sine of $x$.

{\color{blue}
\begin{verbatim}
sin(pi/4)
\end{verbatim}
}

\noindent
$\displaystyle \frac{1}{2^{1/2}}$

\subsection*{sinh($x$)}

Returns the hyperbolic sine of $x$.

{\color{blue}
\begin{verbatim}
circexp(sinh(x))
\end{verbatim}
}

\noindent
$\displaystyle -\tfrac{1}{2}\exp(-x)+\tfrac{1}{2}\exp(x)$

\subsection*{sqrt($x$)}

Returns the square root of $x$.

{\color{blue}
\begin{verbatim}
sqrt(10!)
\end{verbatim}
}

\noindent
$\displaystyle 720\; 7^{1/2}$

\subsection*{stop}

In a script, it does what it says.

\subsection*{sum($i,j,k,f$)}

For $i$ equals $j$ through $k$ evaluate $f$.
Returns the sum of all $f$.

{\color{blue}
\begin{verbatim}
sum(j,1,5,x^j)
\end{verbatim}
}

\noindent
$\displaystyle x^5+x^4+x^3+x^2+x$

\bigskip
\noindent
The original value of $i$ is restored after {\tt sum} completes.
If symbol {\tt i} is used for index variable $i$
then the imaginary unit is overridden in the scope of {\tt sum}.

\subsection*{sum($y$)}

Returns the sum of components of $y$.

{\color{blue}
\begin{verbatim}
y = (1,2,3,4)
sum(y)
\end{verbatim}
}

\noindent
$10$

\subsection*{tan($x$)}

Returns the tangent of $x$.

{\color{blue}
\begin{verbatim}
simplify(tan(x) - sin(x)/cos(x))
\end{verbatim}
}

\noindent
0

\subsection*{tanh($x$)}

Returns the hyperbolic tangent of $x$.

{\color{blue}
\begin{verbatim}
circexp(tanh(x))
\end{verbatim}
}

\noindent
$\displaystyle -\frac{1}{\exp(2x)+1}+\frac{\exp(2x)}{\exp(2x)+1}$

\subsection*{test($a,b,c,d,\ldots$)}

If argument $a$ is true (nonzero) then $b$ is returned, else if $c$ is true then $d$ is returned, etc.
If the number of arguments is odd then the final argument is returned if all else fails.
Expressions can include the relational operators
\verb$=$,
\verb$==$,
\verb$<$,
\verb$<=$,
\verb$>$,
\verb$>=$.
Use the
\verb$not$
function to test for inequality.
(The equality operator
\verb$==$
is available for contexts in which
\verb$=$
is the assignment operator.)

{\color{blue}
\begin{verbatim}
A = 1
B = 1
test(A=B,"yes","no")
\end{verbatim}
}

\noindent
yes

\subsection*{trace}

Set {\tt trace=1} in a script to print the script as it is evaluated.
Useful for debugging.

{\color{blue}
\begin{verbatim}
trace = 1
\end{verbatim}
}

\noindent
Note:
The
\verb$contract$
function is used to obtain the trace of a matrix.

\subsection*{transpose($a,i,j$)}

Returns the transpose of tensor $a$ with respect to indices $i$ and $j$.
If $i$ and $j$ are omitted then 1 and 2 are used.
Hence a matrix can be transposed with a single argument.

{\color{blue}
\begin{verbatim}
A = ((a,b),(c,d))
transpose(A)
\end{verbatim}
}

\noindent
$\displaystyle
\begin{bmatrix}
a & c
\\[1ex]
b & d
\end{bmatrix}
$

\bigskip
\noindent
Note:
The argument list can be extended for multiple transpose operations.
Arguments are evaluated from left to right.
For example,
\verb$transpose(A,1,2,2,3)$
is equivalent to
\verb$transpose(transpose(A,1,2),2,3)$

\subsection*{tty}

Set \verb$tty=1$ to show results in string format.
Set \verb$tty=0$ to turn off.
Can be useful when displayed results exceed window size.

{\color{blue}
\begin{verbatim}
tty = 1
(x + 1)^2
\end{verbatim}
}

\noindent
\verb$x^2 + 2 x + 1$

\subsection*{unit($n$)}

Returns an $n$ by $n$ identity matrix.

{\color{blue}
\begin{verbatim}
unit(3)
\end{verbatim}
}

\noindent
$\displaystyle
\begin{bmatrix}
1 & 0 & 0
\\[1ex]
0 & 1 & 0
\\[1ex]
0 & 0 & 1
\end{bmatrix}
$

\subsection*{zero($i,j,\ldots$)}

Returns a null tensor with dimensions $i$, $j$, etc.
Useful for creating a tensor and then setting component values.

{\color{blue}
\begin{verbatim}
A = zero(3,3)
for(k,1,3,A[k,k]=k)
A
\end{verbatim}
}

\noindent
$\displaystyle
A=
\begin{bmatrix}
1 & 0 & 0
\\[1ex]
0 & 2 & 0
\\[1ex]
0 & 0 & 3
\end{bmatrix}
$

\newpage

\section{Tricks}

\begin{enumerate}

\item
Use \verb$==$ to test for equality.
In effect, \verb$A==B$ is equivalent to \verb$simplify(A-B)==0$.

\item
In a script, line breaking is allowed where the scanner needs something to complete an expression.
For example, the scanner will automatically go to the next line after an operator.

\item
Setting \verb$trace=1$ in a script causes each line to be printed just before it is evaluated.
Useful for debugging.

\item
The last result is stored in symbol \verb$last$.

\item
Use \verb$contract(A)$ to get the mathematical trace of matrix $A$.

\item
Use \verb$binding(s)$ to get the unevaluated binding of symbol $s$.

\item
Use \verb$s=quote(s)$ to clear symbol $s$.

\item
Use \verb$float(pi)$ to get the floating point value of $\pi$.
Set \verb$pi=float(pi)$ to evaluate expressions with a numerical value for $\pi$.
Set \verb$pi=quote(pi)$ to make $\pi$ symbolic again.

\item
Assign strings to unit names so they are printed normally.
For example, setting \verb$meter="meter"$ causes the symbol \verb$meter$
to be printed as meter instead of $m_{eter}$.

\item
Use \verb$expsin$ and \verb$expcos$ instead of \verb$sin$ and \verb$cos$.
Trigonometric simplifications occur automatically when exponentials are used.

\end{enumerate}
\end{document}
