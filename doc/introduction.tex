\input{preamble}

\begin{center}
Eigenmath Manual

\bigskip
9634295@gmail.com
\end{center}

\section*{Introduction}

In the following examples, user input is shown in blue.
Results are shown in black.

\bigskip
Example 1. Compute $212^{17}$.

{\color{blue}
\begin{verbatim}
212^17
\end{verbatim}}

$3529471145760275132301897342055866171392$

\bigskip
Example 2. Compute $212^{17}$ and save as $N$,
then show the value of $N$.

{\color{blue}
\begin{verbatim}
N = 212^17
N
\end{verbatim}}

$N=3529471145760275132301897342055866171392$

\bigskip
Example 3. Compute the 17th root of $N$.

{\color{blue}
\begin{verbatim}
N^(1/17)
\end{verbatim}}

$212$

\iffalse
\bigskip
Note: The above examples were inspired by the following passage from
Vladimir Nabokov's autobiography ``Speak, Memory.''

\begin{quote}
A foolish tutor had explained logarithms to me much too early, and I had
read (in a British publication, the {\it Boy's Own Paper}, I believe)
about a certain Hindu calculator who in exactly two seconds could find the
seventeenth root of, say,
3529471145760275132301897342055866171392
(I am not sure I have got this right; anyway the root was 212).
\end{quote}
\fi

\end{document}
