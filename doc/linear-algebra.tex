\input{preamble}

\section*{Linear algebra}

\verb$dot(a,b,...)$ returns the inner product of vectors, matrices,
and higher rank tensors.
Also known as the matrix product.
Arguments are evaluated from right to left
for optimum efficiency when the last argument is a vector.

\bigskip

Example 1. Compute the product $AX$ for
\begin{equation*}
A=\begin{pmatrix}a_{11}&a_{12}\\a_{21}&a_{22}\end{pmatrix},
\quad
X=\begin{pmatrix}x_1\\x_2\end{pmatrix}
\end{equation*}

{\color{blue}
\begin{verbatim}
A = ((a11,a12),(a21,a22))
X = (x1,x2)
dot(A,X)
\end{verbatim}
}

$\displaystyle
\begin{bmatrix}
a_{11}x_1+a_{12}x_2
\\[1ex]
a_{21}x_1+a_{22}x_2
\end{bmatrix}
$

\bigskip

Example 2. Solve for vector $X$ in $AX=B$.

{\color{blue}
\begin{verbatim}
A = ((3,7),(1,-9))
B = (16,-22)
X = dot(inv(A),B)
X
\end{verbatim}
}

$\displaystyle
X=
\begin{bmatrix}
-\frac{5}{17}
\\[1ex]
\frac{41}{17}
\end{bmatrix}
$

\bigskip

Example 3. Show that
\begin{equation*}
A^{-1}=\frac{\operatorname{adj}A}{\operatorname{det}A}
\end{equation*}

{\color{blue}
\begin{verbatim}
A = ((a,b),(c,d))
inv(A) == adj(A) / det(A)
\end{verbatim}
}

$1$

\iffalse

\bigskip

Square brackets are used for component access.
Index numbering starts with 1.

{\color{blue}
\begin{verbatim}
A = ((a,b),(c,d))
A[1,2] = -A[1,1]
A
\end{verbatim}
}

$\displaystyle
\begin{bmatrix}
a & -a
\\[1ex]
c & d
\end{bmatrix}
$

\bigskip

Sometimes a calculation will be simpler if it can be reorganized to use
\verb$adj$ instead of \verb$inv$.
The main idea is to try to prevent the determinant from appearing as a
divisor.
For example, suppose for matrices $A$ and $B$ you want to show that
\begin{equation*}
{A}-{B}^{-1}=0
\end{equation*}

Depending on the complexity of $\mathop{\rm det}B$, the software
may not be able to find a simplification that yields zero.
A trick is to multiplying by $\operatorname{det}B$ and try
\begin{equation*}
A\operatorname{det}B-\operatorname{adj}B=0
\end{equation*}

\fi

\end{document}
